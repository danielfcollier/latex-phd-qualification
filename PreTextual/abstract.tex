\chapter*{Abstract}

\!\!\!\!\!\!\!\!\!\!\!\!\!\!\!\!\!\!\!

The development and the quest for improvements can be considered important factors to human technological achievements. The sustainable employment of resources and power sources have become a constant concern in a world surrounded by technology. In the field of power processing through systems such as ac and dc, three-phase rectifiers with power factor correction are tipically required due to their ability to improve power quality and performance indices, such as volume reduction and efficiency maximization. Energy conversion applications of ac to dc are best suited for boost topologies, where, among them, the three-phase $\Delta$-switch rectifier ($\Delta$VSR) is the most well suited where a two-level boost type three-phase rectifier is required. This topology and its characteristcs have being employed to improve the performance of others conversion systems and, among them, this work focuses to contributions to the topology to simultaneous operations of multiple rectifiers connected through a multi-interphase transformer (multistate switching cells), named MLMSR-$n\Delta$VSR. This type of rectifier allows well balanced current distribution, frequency multiplication, generation of multiple levels at the input voltage, reduced common mode emissions, among others factors that contribute to its employment where an optimized design for volume reduction or efficiency maximization might be required. This works presents a review of the state-of-the-art of the $\Delta$VSR and initial studies through numerical simulations with new modulation strategies of the MLMSR-$n\Delta$VSR, which show new possibilities and verify the viability to contribute with the stutdy of the presented topology.

\vspace{5mm}

\begin{flushleft}

\small{Keywords: three-phase pwm rectifiers; multilevel rectifiers; multi-state switching cells; multi-interphase transformer; power factor correction.}

\end{flushleft}